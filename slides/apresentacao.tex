\documentclass{beamer}

\mode<presentation>{
\usetheme{Dresden}
\setbeamercovered{transparent}
\usecolortheme{lsc}
}

\mode<handout>{
  % tema simples para ser impresso
  \usepackage[bar]{beamerthemetree}
  % Colocando um fundo cinza quando for gerar transparências para serem impressas
  % mais de uma transparência por página
  \beamertemplatesolidbackgroundcolor{black!5}
}

\usepackage{amsmath,amssymb}
\usepackage[brazil]{varioref}
\usepackage[english,brazil]{babel}
\usepackage[utf8]{inputenc}
%\usepackage[latin1]{inputenc}
\usepackage{graphicx}
\usepackage{listings}
\usepackage{url}
\usepackage{colortbl}
\usepackage[ruled, linesnumbered]{algorithm2e}
\usepackage{amsmath}
\usepackage{hyperref}

\newcommand\Fontvi{\fontsize{6}{10}\selectfont}

\beamertemplatetransparentcovereddynamic

%%%%%%%%%%%%%%%%%%%%%%%%%%%%%%% 
% BIBLIOGRAPHIC REFERENCES
%%%%%%%%%%%%%%%%%%%%%%%%%%%%%%%
\usepackage[sorting=none]{biblatex}
\addbibresource{references.bib}

\title[Atari Games with Deep Q Network]{Atari Games with Deep Q Network}
\author[Niklaus Geisser, Joan Gerard]{%
  Niklaus Geisser\inst{1}\\
  Joan Gerard\inst{1}}
  \institute[ULB]{
  \inst{1}%
     Universit\'e Libre de Bruxelles}

\date{March 20, 2020}

\AtBeginSection[]{
  \begin{frame}<beamer>
    \frametitle{Table of Contents}
    {\footnotesize
    \tableofcontents[currentsection,currentsubsection]
    }
  \end{frame}
}

\begin{document}

\begin{frame}
\titlepage
\end{frame}

\begin{frame}
\frametitle{Table of Contents}
 {\footnotesize
\tableofcontents
}
\end{frame}


\section{What is Deep Reinforcement Learning?}

\frame{
	\frametitle{Reinforcement Learning (RL) Review}
	
	\begin{itemize}
		\item It is the science of decision making.
		\item It involves no supervisor and only a \textbf{reward signal} is used for an agent to determine if they are doing well or not. 
		\item A \textbf{policy}, $\pi$, is a mapping from perceived states of the environment to actions to be taken when in those states.
		\item A \textbf{state}, $s$ is a concrete and immediate situation in which the agent finds itself.
		\item A \textbf{Q function}, also called a state-action value function, specifies how good an action $a$ is in the state $s$.
	\end{itemize}
}

\pgfdeclareimage[height=3cm]{FRAMEWORK}{figs/framework-rl.png}
\pgfdeclareimage[height=4.5cm]{DEEPQ}{figs/deep_reinforcement_learning.PNG}

\frame {
	\frametitle{Reinforcement Learning (RL) Review}
	\begin{itemize}
		\item The \textbf{value of a state} is the total amount of reward an agent can expect to accumulate over the future, starting from that state.
		\item Each \textbf{action} the agent makes affects the next data it receives.
		\item \textbf{Q-value} is the long-term return of an action taking action $a$ under policy $\pi$ from the current state $s$
	\end{itemize}
	\centering
	\pgfuseimage{FRAMEWORK}

}

\frame{
	\frametitle{Deep Reinforcement Learning}
	\textbf{Definition}\\
	Deep reinforcement learning combines \textbf{deep learning} with a \textbf{reinforcement learning} architecture that enables software-defined agents to learn the best actions possible in virtual environment in order 	to attain their goals.
}

\frame {
	\frametitle{Deep Reinforcement Learning}
	\begin{itemize}
		\item RL: finite number of states with finite number of actions. Exhaustive search through all (state, action) pairs to find optimal Q-value: $Q^*(s,a)$.
		\item DRL: very large amount of states, each with lots of actions to try. Approximate Q function with some parameter $\theta$ such that $Q(s, a, \theta) \approx Q^*(s, a)$
	\end{itemize}
}

\frame {
	\frametitle{Deep Reinforcement Learning}
	\begin{itemize}
		\item RL: finite number of states with finite number of actions. Exhaustive search through all (state, action) pairs to find optimal Q-value: $Q^*(s,a)$.
		\item DRL: very large amount of states, each with lots of actions to try. Approximate Q function with some parameter $\theta$ such that $Q(s, a, \theta) \approx Q^*(s, a)$
	\end{itemize}
}
\frame {
	\frametitle{Deep Reinforcement Learning}
	\begin{center}
	   	\pgfuseimage{DEEPQ}
	\end{center}

	
}

\section{Deep Q Networks (DQN)}

\subsection{Update Rule}
\frame{
	\frametitle{Deep Q Networks: Update Rule}
	
	\begin{itemize}
		\item It uses a neural network with weights $\theta$ to approximate the Q-values for all possible actions in each state. It is called \textbf{Q-network}.
		\item Q learning update rule: 
		\begin{equation}
			Q(s,a) = Q(s,a) + \alpha (r + \gamma maxQ(s^\prime, a^\prime) - Q(s,a))
		\end{equation}
		\item $Q(s,a)$ is the current Q-value.
		\item $\alpha$ is the learning rate.
		\item $r$ is the reward for taking action $a$ when in state $s$.
		\item $\gamma$ is the discount rate.
		\item $max Q(s^\prime, a^\prime)$ is the maximum future expected reward given the new state $s^\prime$ and all possible actions at the new state. 
	\end{itemize}
}

\subsection{Target and Predicted value}
\frame {
	\frametitle{Deep Q Networks: Target and Predicted value}
	\centering
	$Q(s,a) = Q(s,a) + \alpha (r + \gamma maxQ(s^\prime, a^\prime) - Q(s,a))$

	\begin{itemize}
		\item $r + \gamma maxQ(s^\prime, a^\prime)$ is the target value. 
		\item Q(s,a) is the predicted value.
		\item Minimize value by learning an optimal policy. 
		
	\end{itemize}
}



\subsection{Loss Function}

\frame {
	\frametitle{Deep Q Networks: Loss Function}
	\begin{itemize}
		\item Minimize the weights $\theta$.
		\begin{equation}
			Loss = (y_i - Q(s, a; \theta)) ^ 2
		\end{equation}
		\item Where $y_i = r + \gamma max_{a^\prime}Q(s^\prime, a^\prime; \theta)$.
		\item Update the weights and minimize the loss through gradient descent.
	\end{itemize}
}
















\section{Architecture of DQN}

\frame{
	\frametitle{Second Part}
	\begin{itemize}
		\item First Point
		\item Second Point
		
	\end{itemize}
}

\subsection{Convolutional Neural Network}
\subsection{Experience Replay}
\subsection{Target Network}
\subsection{Clipping Rewards}


\section{Building it!}

\pgfdeclareimage[height=6cm]{DIAGRAM}{figs/diagram.png}

\frame{
	\frametitle{Main Idea}
	%Maybe show a diagram like this one illustrating the main idea of how it works%
	\centering
	\pgfuseimage{DIAGRAM}
}

\subsection{Libraries} %list the most important libraries we're using%
	%special mention to GYM this one since is the one that allows us to integrate the program with the Atari console%
\pgfdeclareimage[height=6cm]{GYM}{figs/gym.png}
\pgfdeclareimage[height=1cm]{TF}{figs/tf.png}

\frame{
	\frametitle{Libraries}
	\begin{columns}
		\column{6cm}
		\pgfuseimage{GYM}
		\column{5cm}
		\pgfuseimage{TF}
		\begin{itemize}
			\item Gym emulates:
			\begin{itemize}
				\item Atari games
				\item Box2D simulator
				\item Classic control: Cart Pole
				\item MuJoCo: physics simulator
				\item Robotics: Fetch and Push
			\end{itemize}
		\end{itemize}
		
	\end{columns}
}
	
\subsection{Experiments}
\frame{
	\frametitle{Experiments: Parameters} 
	\begin{itemize}
		\item Atari Games: Space Invader, Breakout
		\item Decaying epsilon-greedy policy starting at 0.05.
		\item Number of epochs: 20000.
		\item Learning rate $\alpha = 0.001$.
		\item Discount factor $\gamma = 0.97$.
	\end{itemize}
}

\section{Results}
% show a graph indicating the reward at each epoch


\section{References}

\frame{
	\frametitle{References}
	\begin{itemize}
		\item Sudharsan Ravichandiran. (2018). Hands-On Reinforcement Learning with Python.
		\item Sutton, R. S., Barto, A. G. (2018 ). Reinforcement Learning: An Introduction. The MIT Press. 
		\item Ryan Wong. (2018). Reinforcement Learning: An Introduction to the Concepts, Applications and Code.
		\item Chris Nicholson. (2018). A Beginner's Guide to Deep Reinforcement Learning.
	\end{itemize}
}

\end{document}